%!TEX TS-program = xelatex
%!TEX options = -aux-directory=Debug -shell-escape -file-line-error -interaction=nonstopmode -halt-on-error -synctex=1 "%DOC%"
\documentclass{article}
\input{LaTeX-Submodule/template.tex}

% Additional packages & macros

% Header and footer
\newcommand{\unitName}{Partial Differential Equations}
\newcommand{\unitTime}{Semester 2, 2022}
\newcommand{\unitCoordinator}{Dr Michael Dallaston}
\newcommand{\documentAuthors}{Tarang Janawalkar}

\fancyhead[L]{\unitName}
\fancyhead[R]{\leftmark}
\fancyfoot[C]{\thepage}

% Copyright
\usepackage[
    type={CC},
    modifier={by-nc-sa},
    version={4.0},
    imagewidth={5em},
    hyphenation={raggedright}
]{doclicense}

\date{}

\begin{document}
%
\begin{titlepage}
    \vspace*{\fill}
    \begin{center}
        \LARGE{\textbf{\unitName}} \\[0.1in]
        \normalsize{\unitTime} \\[0.2in]
        \normalsize\textit{\unitCoordinator} \\[0.2in]
        \documentAuthors
    \end{center}
    \vspace*{\fill}
    \doclicenseThis
    \thispagestyle{empty}
\end{titlepage}
\newpage
%
\tableofcontents
\newpage
%
\section{Fourier Series}
\begin{definition}[Fourier series expansion]
    The \textbf{Fourier series expansion} of \(f\) represents \(f\) by a periodic function on an interval, using trigonometric (sine and cosine) terms.

    Suppose a function \(f\left( x \right)\) is defined on an interval \(\interval{-L}{L}\), then
    the Fourier series expansion of \(f\) is given by:
    \begin{equation}\label{eq:fourier}
        f_F\left( x \right) = a_0 + \sum_{n = 1}^\infty a_n \cos{\left( \frac{n \pi x}{L} \right)} + \sum_{n = 1}^\infty b_n \sin{\left( \frac{n \pi x}{L} \right)}
    \end{equation}
    so that \(f = f_F\) on \(\interval{-L}{L}\). Here we cannot be certain about the equailty \(f = f_F\) for all \(x\) as \(f_F\) is perdioc and the convergence properties of the infinite sum are not known.
\end{definition}
Before attempting to determine the coefficients \(a_n\) and \(b_n\) for \(n \geq 1\), we must first evaluate certain useful integral relationships involving trigonometric functions.
\subsection{Integral Relationships}
\subsubsection{Sine and Cosine}
For \(n \in \N\):
\begin{align}
    \int_{-L}^L \cos{\left( \frac{n \pi x}{L} \right)} \odif{x} & = \frac{L}{n \pi} \left[ \sin{\left( \frac{n \pi x}{L} \right)} \right]_{-L}^L            \nonumber \\
                                                                & = \frac{L}{n \pi} \left[ \sin{\left( n \pi \right)} - \sin{\left( -n \pi \right)} \right] \nonumber \\
                                                                & = \frac{L}{n \pi} \left[ 0 - 0 \right]                                                    \nonumber \\
                                                                & = 0. \label{eq:cosine}
\end{align}
\begin{align}
    \int_{-L}^L \sin{\left( \frac{n \pi x}{L} \right)} \odif{x} & = -\frac{L}{n \pi} \left[ \cos{\left( \frac{n \pi x}{L} \right)} \right]_{-L}^L           \nonumber \\
                                                                & = \frac{L}{n \pi} \left[ \cos{\left( n \pi \right)} - \cos{\left( -n \pi \right)} \right] \nonumber \\
                                                                & = \frac{L}{n \pi} \left[ 1 - 1 \right]                                                    \nonumber \\
                                                                & = 0. \label{eq:sin}
\end{align}
\subsubsection{Combinations of Sine and Cosine}
% \begin{align*}
%     f\left( x \right) \cos{\left( \frac{m \pi x}{L} \right)}                      & = \begin{aligned}[t]
%                                                                                           a_0 \cos{\left( \frac{m \pi x}{L} \right)} & + \sum_{n = 1}^\infty a_n \cos{\left( \frac{n \pi x}{L} \right)} \cos{\left( \frac{m \pi x}{L} \right)} \\
%                                                                                                                                      & + \sum_{n = 1}^\infty b_n \sin{\left( \frac{n \pi x}{L} \right)} \cos{\left( \frac{m \pi x}{L} \right)}
%                                                                                       \end{aligned}                                           \\
%     \int_{-L}^L f\left( x \right) \cos{\left( \frac{m \pi x}{L} \right)} \odif{x} & = \begin{aligned}[t]
%                                                                                           a_0 \int_{-L}^L \cos{\left( \frac{m \pi x}{L} \right)} \odif{x} & + \sum_{n = 1}^\infty a_n \int_{-L}^L \cos{\left( \frac{n \pi x}{L} \right)} \cos{\left( \frac{m \pi x}{L} \right)} \odif{x} \\
%                                                                                                                                                           & + \sum_{n = 1}^\infty b_n \int_{-L}^L \sin{\left( \frac{n \pi x}{L} \right)} \cos{\left( \frac{m \pi x}{L} \right)} \odif{x}
%                                                                                       \end{aligned}
% \end{align*}
Recall the Werner formulas:
\begin{gather*}
    2 \cos{\left( \alpha \right)} \cos{\left( \beta \right)} = \cos{\left( \alpha - \beta \right)} + \cos{\left( \alpha + \beta \right)} \\
    2 \sin{\left( \alpha \right)} \sin{\left( \beta \right)} = \cos{\left( \alpha - \beta \right)} - \cos{\left( \alpha + \beta \right)} \\
    2 \sin{\left( \alpha \right)} \cos{\left( \beta \right)} = \sin{\left( \alpha - \beta \right)} + \sin{\left( \alpha + \beta \right)}
\end{gather*}
For \(n,\: m \in \N\),

\underline{Product of two cosine functions:}
\begin{equation*}
    \int_{-L}^L \cos{\left( \frac{n \pi x}{L} \right)} \cos{\left( \frac{m \pi x}{L} \right)} \odif{x} = \frac{1}{2} \int_{-L}^L \cos{\left( \frac{\left( n - m \right) \pi x}{L} \right)} \odif{x} + \frac{1}{2} \int_{-L}^L \cos{\left( \frac{\left( n + m \right) \pi x}{L} \right)} \odif{x}
\end{equation*}
As \(n + m \in \N\), the second integral term will evaluate to \(0\) due to \hyperref[eq:cosine]Equation~{\ref{eq:cosine}}.
For the first integral term, \(n - m \in \N\), except when \(n = m\) which results in \(\cos{\left( 0 \right)} = 1\).
Hence
\begin{equation*}
    \int_{-L}^L \cos{\left( \frac{n \pi x}{L} \right)} \cos{\left( \frac{m \pi x}{L} \right)} \odif{x} = \begin{cases}
        0, & n \neq m \\
        L, & n = m
    \end{cases}
\end{equation*}
\underline{Product of two sine functions:}
\begin{equation*}
    \int_{-L}^L \sin{\left( \frac{n \pi x}{L} \right)} \sin{\left( \frac{m \pi x}{L} \right)} \odif{x} = \frac{1}{2} \int_{-L}^L \cos{\left( \frac{\left( n - m \right) \pi x}{L} \right)} \odif{x} - \frac{1}{2} \int_{-L}^L \cos{\left( \frac{\left( n + m \right) \pi x}{L} \right)} \odif{x}
\end{equation*}
By the same argument,
\begin{equation*}
    \int_{-L}^L \sin{\left( \frac{n \pi x}{L} \right)} \sin{\left( \frac{m \pi x}{L} \right)} \odif{x} = \begin{cases}
        0, & n \neq m \\
        L, & n = m
    \end{cases}
\end{equation*}
\underline{Product of sine and cosine functions:}
\begin{equation*}
    \int_{-L}^L \sin{\left( \frac{n \pi x}{L} \right)} \cos{\left( \frac{m \pi x}{L} \right)} \odif{x} = \frac{1}{2} \int_{-L}^L \sin{\left( \frac{\left( n - m \right) \pi x}{L} \right)} \odif{x} + \frac{1}{2} \int_{-L}^L \sin{\left( \frac{\left( n + m \right) \pi x}{L} \right)} \odif{x}
\end{equation*}
Similary, \(n + m \in \N\) results in \(0\) for the second integral term, \(n - m \in \N\) also results in \(0\) for the first term, and when \(n = m\), as \(\sin{\left( 0 \right)} = 0\), the first term is always \(0\).
Therefore
\begin{equation*}
    \int_{-L}^L \sin{\left( \frac{n \pi x}{L} \right)} \cos{\left( \frac{m \pi x}{L} \right)} \odif{x} = 0
\end{equation*}
\subsection{Coefficients of the Fourier Series}
For \(a_0\) consider integrating \hyperref[eq:fourier]Equation~{\ref{eq:fourier}} from \(-L\) to \(L\).
\begin{align*}
    \int_{-L}^L f\left( x \right) \odif{x} & = \int_{-L}^L a_0 \odif{x} + \sum_{n = 1}^\infty a_n \int_{-L}^L \cos{\left( \frac{n \pi x}{L} \right)} \odif{x} + \sum_{n = 1}^\infty b_n \int_{-L}^L \sin{\left( \frac{n \pi x}{L} \right)} \odif{x} \\
    \int_{-L}^L f\left( x \right) \odif{x} & = 2 a_0 L                                                                                                                                                                                              \\
    a_0                                    & = \frac{1}{2L} \int_{-L}^L f\left( x \right) \odif{x}
\end{align*}
so that \(a_0\) represents the average value of \(f\) on \(\interval{-L}{L}\).

For coefficients \(a_m\), multiply the equation by \(\cos{\left( \frac{m \pi x}{L} \right)}\) before integrating.
\begin{align*}
    f\left( x \right) \cos{\left( \frac{m \pi x}{L} \right)}                      & = \begin{aligned}[t]
                                                                                          a_0 \cos{\left( \frac{m \pi x}{L} \right)} & + \sum_{n = 1}^\infty a_n \cos{\left( \frac{n \pi x}{L} \right)} \cos{\left( \frac{m \pi x}{L} \right)} \\
                                                                                                                                     & + \sum_{n = 1}^\infty b_n \sin{\left( \frac{n \pi x}{L} \right)} \cos{\left( \frac{m \pi x}{L} \right)}
                                                                                      \end{aligned}                                                                                        \\
    \int_{-L}^L f\left( x \right) \cos{\left( \frac{m \pi x}{L} \right)} \odif{x} & = \begin{aligned}[t]
                                                                                          a_0 \cancelto{0}{\int_{-L}^L \cos{\left( \frac{m \pi x}{L} \right)} \odif{x}} & + \sum_{n = 1}^\infty a_n \int_{-L}^L \cos{\left( \frac{n \pi x}{L} \right)} \cos{\left( \frac{m \pi x}{L} \right)} \odif{x}               \\
                                                                                                                                                                        & + \sum_{n = 1}^\infty b_n \cancelto{0}{\int_{-L}^L \sin{\left( \frac{n \pi x}{L} \right)} \cos{\left( \frac{m \pi x}{L} \right)} \odif{x}}
                                                                                      \end{aligned} \\
    \int_{-L}^L f\left( x \right) \cos{\left( \frac{m \pi x}{L} \right)} \odif{x} & = a_m L                                                                                                                                                                                                                     \\
    a_m                                                                           & = \frac{1}{L} \int_{-L}^L f\left( x \right) \cos{\left( \frac{m \pi x}{L} \right)} \odif{x}
\end{align*}

For coefficients \(b_m\), multiply the equation by \(\sin{\left( \frac{m \pi x}{L} \right)}\) before integrating.
\begin{align*}
    f\left( x \right) \sin{\left( \frac{m \pi x}{L} \right)}                      & = \begin{aligned}[t]
                                                                                          a_0 \sin{\left( \frac{m \pi x}{L} \right)} & + \sum_{n = 1}^\infty a_n \cos{\left( \frac{n \pi x}{L} \right)} \sin{\left( \frac{m \pi x}{L} \right)} \\
                                                                                                                                     & + \sum_{n = 1}^\infty b_n \sin{\left( \frac{n \pi x}{L} \right)} \sin{\left( \frac{m \pi x}{L} \right)}
                                                                                      \end{aligned}                                                                                                      \\
    \int_{-L}^L f\left( x \right) \sin{\left( \frac{m \pi x}{L} \right)} \odif{x} & = \begin{aligned}[t]
                                                                                          a_0 \cancelto{0}{\int_{-L}^L \sin{\left( \frac{m \pi x}{L} \right)} \odif{x}} & + \sum_{n = 1}^\infty a_n \cancelto{0}{\int_{-L}^L \cos{\left( \frac{n \pi x}{L} \right)} \sin{\left( \frac{m \pi x}{L} \right)} \odif{x}} \\
                                                                                                                                                                        & + \sum_{n = 1}^\infty b_n \int_{-L}^L \sin{\left( \frac{n \pi x}{L} \right)} \sin{\left( \frac{m \pi x}{L} \right)} \odif{x}
                                                                                      \end{aligned} \\
    \int_{-L}^L f\left( x \right) \sin{\left( \frac{m \pi x}{L} \right)} \odif{x} & = b_m L                                                                                                                                                                                                                                   \\
    b_m                                                                           & = \frac{1}{L} \int_{-L}^L f\left( x \right) \sin{\left( \frac{m \pi x}{L} \right)} \odif{x}
\end{align*}
To summarise,
\begin{gather*}
    a_0 = \frac{1}{2L} \int_{-L}^L f\left( x \right) \odif{x} \\
    a_m = \frac{1}{L} \int_{-L}^L f\left( x \right) \cos{\left( \frac{m \pi x}{L} \right)} \odif{x} \\
    b_m = \frac{1}{L} \int_{-L}^L f\left( x \right) \sin{\left( \frac{m \pi x}{L} \right)} \odif{x}
\end{gather*}
for \(m \in \N\).
\begin{definition}[Piecewise smooth]
    A function \(f : \interval{a}{b} \to \R\), is \textbf{piecewise smooth} if each component \(f_i\) of \(f\) has a \underline{bounded derivative} \(f_i'\) which is \underline{continuous everywhere} in \(\interval{a}{b}\), except at
    a finite number of points at which left- and right-sided derivatives exist.
\end{definition}
\begin{theorem}[Convergence of piecewise smooth functions]
    If \(f\) is a periodic piecewise smooth function on \(\interval{-L}{L}\), \(f_F\) will converge to
    \begin{equation*}
        f_F\left( x \right) = \lim_{\epsilon \to 0^{+}} \frac{f\left( x + \epsilon \right) + f\left( x - \epsilon \right)}{2}
    \end{equation*}
    that is, \(f = f_F\), except at discontinuities, where \(f_F\) is equal to the point halfway between the left- and right-hand limits.
\end{theorem}
\begin{corollary}[Dirichlet conditions]
    The Dirichlet conditions provide sufficient conditions for a real-valued function \(f\) to be
    equal to its Fourier series \(f_F\) on \(\interval{-L}{L}\), at each point where \(f\) is continuous.

    The conditions are:
    \begin{enumerate}
        \item \(f\) has a finite number of maxima and minima over \(\interval{-L}{L}\).
        \item \(f\) has a finite number of discontinuities, in each of which the derivative \(f'\) exists
              and does not change sign.
        \item \(\int_{-L}^L \abs*{f\left( x \right)} \odif{x}\) exists.
    \end{enumerate}
\end{corollary}
\begin{definition}[Gibbs phenomenon]
    If \(f_F\) does not converge to \(f\) at discontiniuties \(x_i\), then the \(f_F\) converges
    non-uniformally. For Fourier series this property is known as the \textit{Gibbs phenomenon}.
\end{definition}
\begin{note}
    When \(f\) is non-periodic, \(f_F\) converges to the periodic extension of \(f\).
    The endpoints may converge non-uniformally, corresponding to jump disontinuities in the periodic extension of \(f\).
\end{note}
\subsection{Sine and Cosine Series}
\begin{definition}[Odd function]
    \(f\) is an \textit{odd} function if it satisfies
    \begin{equation*}
        f\left( -x \right) = -f\left( x \right)
    \end{equation*}
\end{definition}
\begin{definition}[Even function]
    \(f\) is an \textit{even} function if it satisfies
    \begin{equation*}
        f\left( -x \right) = f\left( x \right)
    \end{equation*}
\end{definition}
If \(f\) is an odd function on \(\interval{-L}{L}\), then the coefficients corresponding to the cosine terms will be zero.
The Fourier series simplifies to
\begin{equation*}
    f_F = \sum_{n = 1}^\infty b_n \sin{\left( \frac{n \pi x}{L} \right)}
\end{equation*}
where \(b_n = \frac{2}{L} \int_0^L f\left( x \right) \sin{\left( \frac{n \pi x}{L} \right)} \odif{x}\).

Likewise if \(f\) is an even function on \(\interval{-L}{L}\), then the coefficients corresponding to the sine terms will be zero.
The Fourier series simplifies to
\begin{equation*}
    f_F = a_0 + \sum_{n = 1}^\infty a_n \cos{\left( \frac{n \pi x}{L} \right)}
\end{equation*}
where \(a_0 = \frac{1}{L} \int_0^L f\left( x \right) \odif{x}\) and \(a_n = \frac{2}{L} \int_0^L f\left( x \right) \sin{\left( \frac{n \pi x}{L} \right)} \odif{x}\).

These special cases are known as the sine and cosine series expansions respectively, resulting in the \textbf{odd} or \textbf{even} periodic extension of \(f\).
\section{Partial Differential Equations}
A partial differential equation (PDE) is a differential equation that must be solved for an unknown function of at least two independent variables,
where the equation contains partial derivatives of the unknown function.

PDEs are characterised by a number of properties:
\begin{itemize}
    \item The \textbf{order} of the PDE is order of the highest derivative in the equation.
          Furthermore, each independent variable can be described by its own order.
    \item A PDE is \textbf{linear} if it is linear in its unknown function and its derivatives.
          \begin{itemize}
              \item A linear PDE has \textbf{constant} coefficients if the coefficients if the linear terms do not depend on the independent variables, and has \textbf{variable} coefficients otherwise.
              \item A linear PDE is \textbf{homogeneous} if all terms depend on the unknown function, and \textbf{nonhomogeneous} otherwise.
          \end{itemize}
\end{itemize}
\subsection{Initial Boundary Value Problems}
As with ODEs, we can find the general solution to a PDE and then use initial/boundary conditions to solve for arbitrary constants.
The number of conditions for each independent variable depends on the order of that variable in the PDE\@. Problems with initial and boundary
conditions are called \textbf{initial boundary value problems} and are often referred to as \textbf{IBVPs}.
\subsection{Linear Operators and Superposition}
By linearity, we can write a PDE in terms of linear operators. For example, we can write the PDE
\begin{equation*}
    \pdv[order=2]{u}{x} = \pdv{u}{x} + \pdv{u}{t}
\end{equation*}
as
\begin{equation*}
    L\left( u \right) = \pdv[order=2]{u}{x} - \pdv{u}{x} - \pdv{u}{t} \iff L = \pdv[order=2]{}{x} - \pdv{}{x} - \pdv{}{t}.
\end{equation*}
Similary, we can describe initial/boundary conditions as linear or homogeneous.
\begin{theorem}[Superposition]
    If \(u_n\), \(n = 1,\: \dots,\: N\) are solutions to the homogeneous PDE \(L\left( u \right) = 0\), then any linear combination
    of these solutions is a solution to the PDE
    \begin{equation*}
        u = \sum_{n = 1}^N c_n u_n
    \end{equation*}
    where \(c_n\) are constants.
\end{theorem}
The following three sections cover the most common kinds of PDEs and their solutins.
\subsection{Heat Equation}
Consider the temperature \(u\left( x,\: t \right)\) of a 1d metal rod of length \(L\) at an initial temperature \(u_0\left( x \right) = u\left( x,\: 0 \right)\) and boundary conditions \(T_1 = u\left( 0,\: t \right)\) and \(T_2 = u\left( L,\: t \right)\).
If we consider a small section of this rod \(\interval{x_1}{x_2} \in \interval{0}{L}\), then the rate of change of heat \(H\left( x,\: t \right)\) in this section is given by
\begin{align*}
    \text{Rate of change of heat energy} & = \text{Flow in} - \text{Flow out}                    \\
    \int_{x_1}^{x_2} \pdv{H}{t} \odif{x} & = Q\left( x_1,\: t \right) - Q\left( x_2,\: t \right)
\end{align*}
where \(Q\left( x,\: t \right)\) is the heat flux at time \(t\).

By making the following assumptions, we can formulate a relationship for the temperature in the rod at position \(x\) at time \(t\).
\begin{enumerate}
    \item No energy is lost in the rod.
    \item The change in heat energy is proportional to the change in temperature (i.e., no phase changes are present), so that the specific heat equation applies.
          \begin{equation*}
              \adif{H} = \rho c \adif{u} \iff \pdv{H}{t} = \rho c \pdv{u}{t}
          \end{equation*}
          where \(\rho\) is the density of the rod and \(c\) is the specific heat of the rod.
    \item The material of the rod is homogeneous, and Fourier's law of conduction applies.
          \begin{equation*}
              \symbfit{Q} = -\kappa \nabla \symbfit{u} \iff Q = -\kappa \pdv{u}{x}
          \end{equation*}
          where \(Q = Q\left( x,\: t \right)\) is the heat flux at time \(t\), and \(\kappa\) is the thermal conductivity of the rod.
\end{enumerate}
By substituing these assumptions into the above equation, we can write the PDE as
\begin{align*}
    \int_{x_1}^{x_2} \pdv{H}{t} \odif{x}        & = Q\left( x_1,\: t \right) - Q\left( x_2,\: t \right)                               \\
    \int_{x_1}^{x_2} \rho c \pdv{u}{t} \odif{x} & = \left[ -\kappa \pdv{u}{x} \right]_{x_1} - \left[ -\kappa \pdv{u}{x} \right]_{x_2} \\
    \int_{x_1}^{x_2} \rho c \pdv{u}{t} \odif{x} & = \left[ \kappa \pdv{u}{x} \right]_{x_2} - \left[ \kappa \pdv{u}{x} \right]_{x_1}   \\
    \int_{x_1}^{x_2} \rho c \pdv{u}{t} \odif{x} & = \int_{x_1}^{x_2} \pdv*{\kappa \pdv{u}{x}}{x} \odif{x}                             \\
    \rho c \pdv{u}{t}                           & = \pdv*{\kappa \pdv{u}{x}}{x}                                                       \\
    \pdv{u}{t}                                  & = \frac{\kappa}{\rho c} \pdv[order=2]{u}{x}.
\end{align*}
Simplifying the constants, we get
\begin{equation*}
    \pdv{u}{t} = k\pdv[order=2]{u}{x}
\end{equation*}
where \(k\) is the thermal diffusivity of the rod. More generally, we can write the PDE as
\begin{equation*}
    \pdv{\symbfit{u}}{t} = k \Delta{\symbfit{u}}
\end{equation*}
for a multiple spatial dimensions. This PDE is called the heat equation.

The heat equation is first order w.r.t.\ time and second order w.r.t.\ space.
\subsection{Wave Equation}
\begin{equation*}
    \pdv[order=2]{u}{t} = c\pdv[order=2]{u}{x}
\end{equation*}
\subsection{Laplace's Equation}
\begin{equation*}
    \pdv[order=2]{u}{t} = c \nabla^2 u
\end{equation*}
\section{Separation of Variables}
To solve an IBVP, consider the set of solutions \(u_n\) that satisfy the \textit{homogeneous}
PDE and the \textit{homogeneous} boundary conditions. Assume a solution of the form
\begin{equation*}
    u\left( x,\: t \right) = X\left( x \right) T\left( t \right)
\end{equation*}
so that by substitution, the linear homogeneous PDE can be separated into additive terms involving each independant variable.
This equation has the form:
\begin{equation*}
    f_1\left( x,\: X,\: X',\: \dots \right) = f_2\left( t,\: T,\: T',\: \dots \right)
\end{equation*}
Then as each term depends on a different variable, the homogeneous equation is only satisfied if each function \(f_i\) evaluates to a scalar \(\alpha_n\).
Hence
\begin{align*}
    f_1\left( x,\: X,\: X',\: \dots \right) & = \alpha_n  \\
    f_2\left( t,\: T,\: T',\: \dots \right) & = \alpha_n.
\end{align*}
Once the homogeneous equation is separated, we must solve the ODEs that depend on variables with which boundary conditions are
applied. This involves assuming various ranges for \(\alpha_n\), namely, negative, zero or positive, with the aim of
finding non-trivial solutions.

This will then allow us to solve for the remaining ODEs, and use the linearity of the solutions to superpose them for all values of \(n\).
We can then apply any initial conditions to this superposed form.
\subsection{Separation of Variables: Heat Equation}
\begin{align*}
    \pdv{u}{t}               & = k\pdv[order=2]{u}{x}     \\
    \pdv{XT}{t}              & = k\pdv[order=2]{XT}{x}    \\
    XT'                      & = kX''T                    \\
    \frac{1}{k} \frac{T'}{T} & = \frac{X''}{X} = \alpha_n
\end{align*}
This results in the following two ODEs
\begin{align*}
    T' - k \alpha_n T = 0 \\
    X'' - \alpha_n X = 0
\end{align*}
\subsubsection{Spatial Dimension}
\begin{proofcase}{1. \(\alpha_n > 0\)}\let\qed\relax
    \begin{align*}
        m^2 - \alpha_n & = 0                     \\
        m              & = \pm \sqrt{ \alpha_n }
    \end{align*}
    Therefore
    \begin{equation*}
        X_n\left( x \right) = c_1 e^{\sqrt{ \alpha_n } x} + c_2 e^{-\sqrt{ \alpha_n } x}.
    \end{equation*}
    Applying the BCs gives
    \begin{align*}
        X_n\left( 0 \right) & = c_1 + c_2 = 0                                              \\
        X_n\left( L \right) & = c_1 e^{\sqrt{\alpha_n} L} + c_2 e^{-\sqrt{\alpha_n} L} = 0
    \end{align*}
    so that
    \begin{equation*}
        \begin{bmatrix*}
            1 & 1 \\
            e^{\sqrt{\alpha_n} L} & e^{-\sqrt{\alpha_n} L}
        \end{bmatrix*}
        \begin{bmatrix*}
            c_1 \\
            c_2
        \end{bmatrix*}
        =
        \begin{bmatrix*}
            0 \\
            0
        \end{bmatrix*}
    \end{equation*}
    This homogeneous equation has non-trivial solutions iff the determinant is zero.
    \begin{align*}
        \begin{vmatrix*}
            1 & 1 \\
            e^{\sqrt{\alpha_n} L} & e^{-\sqrt{\alpha_n} L}
        \end{vmatrix*} & = 0        \\
        e^{-\sqrt{\alpha_n} L} - e^{\sqrt{\alpha_n} L}                 & = 0 \\
        -2\sinh{\left( \sqrt{\alpha_n} L \right)}                      & = 0 \\
        \alpha_n                                                       & = 0
    \end{align*}
    but as \(\alpha_n > 0\), no solutions exist.
\end{proofcase}
\begin{proofcase}{2. \(\alpha_n = 0\)}\let\qed\relax
    \begin{equation*}
        X_n\left( x \right) = c_1 x + c_2.
    \end{equation*}
    Applying the BCs gives
    \begin{align*}
        X_n\left( 0 \right) & = c_2 = 0                                        \\
        X_n\left( L \right) & = c_1 e^{\sqrt{\alpha_n} L} = 0 \implies c_1 = 0
    \end{align*}
    hence there are no non-trivial solutions as \(X_n \equiv 0\).
\end{proofcase}
\begin{proofcase}{3. \(\alpha_n < 0\)}\let\qed\relax
    \begin{align*}
        m^2 + \alpha_n & = 0                      \\
        m              & = \pm \sqrt{-\alpha_n} i \\
    \end{align*}
    therefore
    \begin{equation*}
        X_n\left( x \right) = c_1 \cos{\left( \sqrt{-\alpha_n} x \right)} + c_2 \sin{\left( \sqrt{-\alpha_n} x \right)}.
    \end{equation*}
    Applying the BCs gives
    \begin{align*}
        X_n\left( 0 \right) & = c_1 = 0                                         \\
        X_n\left( L \right) & = c_2 \sin{\left( \sqrt{-\alpha_n} L \right)} = 0
    \end{align*}
    therefore
    \begin{align*}
        \sqrt{-\alpha_n} L & = n \pi                  \\
        \alpha_n           & = -\frac{n^2 \pi^2}{L^2}
    \end{align*}
    which gives the following family of solutions:
    \begin{equation*}
        X_n\left( x \right) = c_2 \sin{\left( \frac{n \pi}{L} x \right)}
    \end{equation*}
\end{proofcase}
\subsubsection{Time Dimension}
Substituting \(\alpha_n = -\frac{n^2 \pi^2}{L^2}\) into the time ODE gives
\begin{align*}
    m - k \alpha_n & = 0          \\
    m              & = k \alpha_n
\end{align*}
therefore
\begin{equation*}
    T_n\left( t \right) = c_3 e^{-\frac{n^2 \pi^2 k}{L^2} t}
\end{equation*}
\subsubsection{General Solution}
Given these two functions we can solve for \(u_n\) as
\begin{align*}
    u_n\left( x,\: t \right) = A_n \sin{\left( \frac{n \pi}{L} x \right)} e^{-\frac{n^2 \pi^2 k}{L^2} t}
\end{align*}
then by applying superposition, we find the general solution to the PDE\@:
\begin{equation*}
    u\left( x,\: t \right) = \sum_{n = 1}^\infty u_n\left( x,\: t \right) = \sum_{n = 1}^\infty A_n \sin{\left( \frac{n \pi}{L} x \right)} e^{-\frac{n^2 \pi^2 k}{L^2} t}.
\end{equation*}
Applying the initial conditions gives
\begin{align*}
    u_n\left( x,\: 0 \right) = \sum_{n = 1}^\infty A_n \sin{\left( \frac{n \pi}{L} x \right)} = u_0
\end{align*}
so that the coefficients \(A_n\) are given by the Fourier sine coefficients of the initial condition \(u_0\).
Finally, the general solution to the PDE is given by
\begin{equation*}
    u\left( x,\: t \right) = \sum_{n = 1}^\infty A_n \sin{\left( \frac{n \pi}{L} x \right)} e^{-\frac{n^2 \pi^2 k}{L^2} t}, \quad \quad A_n = \frac{2}{L} \int_0^L u_0 \sin{\left( \frac{n \pi}{L} x \right)} \odif{x}.
\end{equation*}
We observe that as time tends to infinity, the solution approaches \(0\) as the exponential term
becomes \(1\). We can therefore approximate the solution at this time with 
\begin{equation*}
    u\left( x,\: \infty \right) \approx A_1 \sin{\left( \frac{\pi}{L} x \right)}.
\end{equation*}
\end{document}